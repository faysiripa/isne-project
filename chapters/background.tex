\chapter{\ifenglish Background Knowledge and Theory\else ทฤษฎีที่เกี่ยวข้อง\fi}

% การทำโครงงาน เริ่มต้นด้วยการศึกษาค้นคว้า ทฤษฎีที่เกี่ยวข้อง หรือ งานวิจัย/โครงงาน ที่เคยมีผู้นำเสนอไว้แล้ว ซึ่งเนื้อหาในบทนี้ก็จะเกี่ยวกับการอธิบายถึงสิ่งที่เกี่ยวข้องกับโครงงาน เพื่อให้ผู้อ่านเข้าใจเนื้อหาในบทถัดๆ ไปได้ง่ายขึ้น

\section{Internet of Things (IoT)}
The Internet of Things (IoT) refers to a network of interconnected devices that can sense, process, and exchange data with minimal human intervention. These devices are widely deployed in applications such as smart homes, healthcare monitoring, environmental sensing, transportation systems, and industrial automation. While IoT enables efficiency and automation, it also introduces security challenges. Many IoT devices are resource-constrained in terms of processing power, memory, and energy supply, making it difficult to implement traditional, computation-heavy cryptographic techniques. As a result, IoT networks are often vulnerable to attacks such as eavesdropping, replay, and message injection.

\section{LoRa Technology}
LoRa (Long Range) is a low-power wide-area network (LPWAN) communication technology that enables devices to transmit data over several kilometers while consuming minimal energy. This makes it suitable for IoT applications where devices must operate on battery power for extended periods. LoRa achieves long-distance communication using Chirp Spread Spectrum (CSS) modulation, which provides robustness against noise and interference.
 However, LoRa’s primary limitation is its lack of built-in security mechanisms. The physical layer itself does not provide strong confidentiality or integrity protection. While LoRaWAN (the higher-layer protocol) introduces some security features, lightweight LoRa implementations—such as those used in simple ESP32 + LoRa projects—are highly vulnerable to interception, spoofing, and key extraction if insecure practices (e.g., hardcoded keys) are used.

 \section{Received Signal Strength Indicator (RSSI)}
 The Received Signal Strength Indicator (RSSI) measures the power level of a received wireless signal, typically expressed in decibels relative to one milliwatt (dBm). In LoRa systems, RSSI is automatically measured at the receiver whenever a packet is received. Normally, RSSI is used to evaluate link quality or assist in adaptive communication strategies.
 In the context of security, RSSI can also be leveraged as a source of entropy for key generation. Since RSSI values fluctuate depending on distance, obstacles, interference, and environmental conditions, they are inherently dynamic and difficult for an attacker to predict without being physically co-located in the communication channel. By using RSSI values as a basis for session key generation, IoT devices can avoid reliance on static, pre-shared keys that are easy to compromise.
\section{ Lightweight Cryptography}
Lightweight cryptography refers to cryptographic techniques designed specifically for devices with limited computational and memory resources. Unlike traditional algorithms such as AES-256 or RSA, which require significant processing power, lightweight algorithms are optimized for efficiency while maintaining an acceptable level of security. Typical strategies include reducing key sizes, minimizing memory overhead, or designing algorithms tailored for 8-bit or 32-bit microcontrollers.
 For IoT applications using devices like the ESP32, lightweight cryptography is essential to balance security with system performance. The challenge is to implement schemes that protect data confidentiality and integrity without exceeding constraints such as 10 KB of program memory or 300 bytes of RAM.
\section{Key Management in IoT}
Key management is one of the most critical aspects of securing IoT networks. Traditional approaches rely on pre-shared static keys, which pose significant risks: once an attacker obtains the key, they can decrypt all subsequent communications. Dynamic key management schemes provide stronger protection by regularly updating session keys.
 One promising approach is to derive session keys from physical-layer metrics, such as RSSI. Since each device independently measures RSSI during communication, keys can be generated locally without transmitting sensitive information over the air. To ensure synchronization, both devices agree on an adjustment interval and use a shared keyword for verification. This method significantly reduces the risk of interception and makes it difficult for attackers to reuse keys, thereby enhancing overall IoT security.
