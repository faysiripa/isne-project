\maketitle
\makesignature

\ifproject
\begin{abstractTH}
เทสเทสเทส
\end{abstractTH}

\begin{abstract}
LoRa (Long Range) is a low-power wireless communication technology widely used in Internet of Things (IoT) applications. These include environmental monitoring, agriculture, and infrastructure systems. While LoRa excels in long-range communication, it lacks native security features. Data transmitted over LoRa can be intercepted, and hardcoded encryption keys on microcontrollers like the ESP32 are vulnerable to extraction via physical access. This poses a serious security threat to the IoT network. 
 
This project presents a secure communication framework for IoT systems using LoRa and MicroPython, focusing on lightweight encryption and key management. Data transmissions are encrypted using the cryptolib module, and session keys are updated based on shared physical metrics such as RSSI values and communication response timing. These metrics are known only to the legitimate communicating devices which  increases resistance to message interception and spoofing. The approach is designed for resource-constrained and physically exposed environments. It provides a practical solution for securing IoT communications without compromising performance or power efficiency.

\end{abstract}

\iffalse
\begin{dedication}
This document is dedicated to all Chiang Mai University students.

Dedication page is optional.
\end{dedication}
\fi % \iffalse

\begin{acknowledgments}
Your acknowledgments go here. Make sure it sits inside the
\texttt{acknowledgment} environment.

\acksign{2020}{5}{25}
\end{acknowledgments}%
\fi % \ifproject

\contentspage

\ifproject
\figurelistpage

\tablelistpage
\fi % \ifproject

% \abbrlist % this page is optional

% \symlist % this page is optional

% \preface % this section is optional
