\chapter{\ifproject%
\ifenglish Experimentation and Results\else การทดลองและผลลัพธ์\fi
\else%
\ifenglish System Evaluation\else การประเมินระบบ\fi
\fi}

\section{Evaluation Metrics}
If implemented, the proposed framework would be evaluated using the following criteria:

\subsection{Performance}
 the time required for key generation, encryption, and decryption on ESP32 devices.

\subsection{Resource Usage}
 memory and CPU consumption, ensuring the design remains lightweight.

\subsection{Security Assessment}
 resistance against common IoT attacks, including replay and man-in-the-middle (MITM) attacks.

 \section{Evaluation Method}

\subsection{Correctness Testing}
Repeatedly transmit verification messages and record the percentage of times the receiver derives the correct key and matches the predefined keyword.

\subsection{Performance Measurement}
Measure execution time of key generation and encryption/decryption functions using built-in timers on ESP32.

\subsection{Resource Usage}
Monitor program memory footprint and RAM usage to ensure they remain under the target limits (~10 KB flash, ~300 bytes RAM).

\subsection{ Security Testing}
Simulate potential attack scenarios such as replaying old packets or attempting to intercept communication, to evaluate whether the dynamic key mechanism prevents message reuse or prediction.

 \section{Evaluation Method}
 The evaluation is expected to show that the RSSI-based key generation approach achieves high correctness in key agreement, maintains lightweight resource usage suitable for ESP32 devices, and provides improved resilience against interception compared to static key methods.